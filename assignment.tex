\documentclass[12pt]{article}
\usepackage[margin=1in]{geometry}

\usepackage[shortlabels]{enumitem}
\setcounter{secnumdepth}{0}

\usepackage{graphicx}
\usepackage{caption}

\usepackage{url}

\usepackage{fancyhdr, lastpage}
\pagestyle{plain}
\fancyhf{}
%
%\rhead{}
%\lhead{}
%\rhead{}
%
\cfoot{Page \thepage{} of \protect\pageref*{LastPage}}

\usepackage{varioref}
\labelformat{equation}{(#1)}

% \usepackage{hyperref} must almost always be LAST \usepackage in
% preamble. Otherwise, you may get strange compilation errors!
\usepackage[colorlinks,linkcolor=blue]{hyperref}
 
\begin{document}
\section{The most common ESL error types}
\begin{enumerate}
\item \url{http://www.englishclub.com/grammar/adjectives-determiners-the-a-an.htm}\label{1}
\item \url{http://en.wikipedia.org/wiki/Article_(grammar)#Definite_article}\label{2}
\item \url{http://grammar.ccc.commnet.edu/grammar/determiners/determiners.htm}\label{3}
\item \url{http://college.cengage.com/devenglish/fawcett/evergreen/7e/students/esl_errors.html}\label{4}
\item \url{http://en.wikipedia.org/wiki/Determiner_(linguistics)}\label{5}
\item \url{http://ezinearticles.com/?Common-Errors-With-Prepositions&id=1081907}\label{7}
\item \url{http://en.wikipedia.org/wiki/List_of_English_prepositions}\label{8}
\item \url{http://ezinearticles.com/?Common-ESL-Errors&id=1809651}\label{9}
\item \url{http://www.ego4u.com/en/cram-up/grammar/tenses}\label{10}
\item \url{http://www.yourdictionary.com/grammar-rules/5-most-common.html}\label{11}
\item \url{http://www.englishpage.com/gerunds/gerund_list.htm}\label{12}
\item \url{http://www.englishpage.com/gerunds/infinitive_list.htm}\label{13}
\item \url{http://www.ling.ohio-state.edu/icall/calico08/gamon-et-al.pdf}

\end{enumerate}

\subsection{1. Noun Related Errors}
\subsubsection{Incorrect or Missing Determiners}
An article error should be identified when:
\begin{enumerate}[a)]
\item an incorrect article is used ({\it a/an} instead of {\it the} or vice versa);
\item an article is missing, but it is required;
\item an unnecessary article appears.
% ({\it e.g.,} with meals, diseases, seasons, institutions, time of day)
\end{enumerate}

Quick review: Articles {\it a} and {\it an} are used in front of nonspecific, singular count nouns ({\it a movie, a cat}). An article {\it the} is used in front of specific singular and plural nouns ({\it the movie, the cats}). Many uncountable nouns require no article at all {\it honesty, news etc}. 
\begin{enumerate}[a)]
\item Incorrect article

{\it In Europe, the First World War started coming to *an/the end after a huge destruction.}

{\it Blogs are considered as *the/a direct way to participate in politics.}
%{\it *A/The capital of France is Paris.} (\ref{1})\\
%{\it Kyiv is *a/the capital of Ukraine.}\\
%{\it I ordered *the/a cup of tea.}\\
%{\it Mary found the dress that she lost.}
%{\it I have found *a/the book that I lost.} (\ref{1})\\
%{\it James Bond ordered *the/a drink.} (\ref{1})

\item Missing article

{\it Command line scripting: PHP script can run without */a server.}

{\it There were no instances of NTR faltering in front of */the camera.}

{\it Hardware errors are one type of errors in */the kernel.}

%{\it I had */a wonderful day.}\\
%{\it There are six apples in */the basket.}\\
%{\it John is looking in */the window now.}
%{\it  There was *?/a beautiful sunrise this morning.} (\ref{4})\\
%{\it We want to buy *?/an  umbrella.} (\ref{1})\\
%{\it There are six eggs in *?/the fridge.} (\ref{1})

\item Unnecessary article

{\it All *the warring nations commenced deliberations.}

{\it However, *the specific defects are not known.}

{\it Many interactive multimedia encyclopedias written by *the people existed before Wikipedia.}

{\it They could not attain a victory.}
%{\it *A/ football is very popular in Europe.}\\
%{\it *The/ breakfast was delicious.}\\
%{\it Children like to be outside in *the/ spring.}
%{\it In *the spring, we like to clean the house.}  (\ref{2})\\
%{\it *The visitors walked in mud.} (\ref{2})\\
%{\it She plays *a badminton and *a basketball.} (\ref{3})
\end{enumerate}

Other determiners including {\it this/that, these/those, each/every, either/neither, some/any, much/many/few/little/several/most, enough etc.} should be corrected if they used incorrectly.  

Quick review of other determiners in English [\ref{3}], [\ref{5}]:
\begin{itemize}[-]
\item Demonstratives: \\
{\it this/that} is used with singular count nouns ({\it this dog}), non-count nouns ({\it that coffee});\\
{\it these/those} is used with plural count nouns ({\it these dogs, those people}).

\item Distributive determiners: \\
{\it each/every} is used with single count nouns ({\it each bottle, every person}).

\item Disjunctive determiners: \\
{\it either/neither} is used with singular count nouns ({\it either hat is fine}).

\item Existential determiners: \\
{\it some/any} is used with singular and plural count nouns, noncount nouns ({\it some car, any new houses}).

\item Degree determiners: \\
{\it e.g., much} is used with noncount nouns ({\it much wine, much furniture});\\
{\it e.g., enough} is used with plural count nouns ({\it enough troubles}), noncount nouns ({\it enough furniture}).
\end{itemize}

{\it Sariyat is *that/a complete policy which is based on Islamic law traditions.}

\subsubsection{Noun Numbers}
Errors involving plural/singular confusion of a noun should be identified as shown in example sentences below:


{\it In terms of population, India is one of the *worlds/world leaders.}
%{\it Vienna is one of the most romantic *city/cities in the world.}

%{\it Learning English takes a *efforts/effort.}

%{\it Research requires *many knowledge/much knowledge.}

%{\it London is one of the most attractive *city/cities in the world.}

%{\it You have to write down all the details of each *things/thing to do.}

%{\it Conversion always takes a lot of *efforts/effort.}

%{\it It requires *many knowledge/much knowledge.}

\subsubsection{Noun of Noun/Word Order}
Errors involving incorrect word order in noun phrases must be identified as shown in example sentences below:

{\it Similarly to other big *cities of India/India cities, Hyderabad Police has one IPS Officer.}

{\it The largest financial *services companies/service companies in the world are in Japan.}
%{\it *Feedback of customers/customer feedback is vital to making a business work.}

%{\it Marry is *a student of university/a university student.}

\subsubsection{Pronoun Errors}
Pronouns must agree in number with the noun (the antecedent) to which they refer {\it e.g., I - me/my, you - yours, he - him/his, she - her, we - our/us, they - their/them}.

{\it America, Britain and Japan landed *its/their armies in Vladovostik.}
 
%{\it Everybody must submit *their/his own code.}

%{\it John said that *himself/he played basketball.}

%{\it Mary ensured *myself/me that I can rely on *herself/her.}

%{\it Everybody must bring *their/his own lunch.} (\ref{11})
%{\it Neither of the girls brought *their/her umbrella.}
%{\it Some of the flowers have lost *its/their petals.}

\subsection{2. Preposition Related Errors}
Prepositions like [\ref{8}] {\it about, at, by, down, for, from, in, into, of, off, on, onto, out, over, to, up, upon, with, within, etc.} are difficult ESL writers as can be seen from example sentences below. In this step you should correct any prepositions that are used incorrectly. 

{\it This is fast */in speed, independent, reliable and diversified platform.}

{\it *For it does not require any compilation.}

{\it As a result of its support *for/in the war.}

{\it He should add Vishakapatnam N.T.D code 0891 *for/to this phone number.}

%{\it We will go to the mountains *on/in March.}

%{\it Students did not pay much attention *on/to the speaker.}

%{\it Professor described the homework *to/for students.}

%{\it It seems ok and I did not pay much attention *on/to it.} (\ref{6})

%{\it Below is my contact, looking forward *your/to your response, thanks!} (\ref{6})

Quick review of prepositions in English [\ref{7}]:
\begin{itemize}[-]
\item {\it by, with}

{\it by} is used to refer to the doer of an action {\it e.g., He was killed by his servant}; \\
{\it with} is used to refer to the instrument with which the action is done {\it e.g., He was killed with an axe}.

\item {\it beside vs. besides}

{\it beside} means ``by the side of" {\it e.g., The house was beside the river};\\
{\it besides} means ``in addition to" {\it e.g., He plays tennis besides basketball and football}.

\item {\it since, for}

{\it since} refers to the starting point of an action and means ``from a particular point of time in the past" {\it e.g, He has been absent since last Monday};\\
{\it for} is used to talk about duration and refers to a period of time {\it e.g., I have been waiting here for two hours}.

\item {\it between vs. among}

{\it between} is used to say that somebody or something is between two or more clearly separate objects {\it e.g., You have to choose between these two options};\\
{\it among} is used with more than two people or things {\it e.g., The British were able to conquer India because the Indian princes quarreled among themselves}.

\item {\it at, in}

{\it in} is generally used to refer to large places - countries, districts, large cities {\it etc. e.g., My brother lives at Mumbai};\\
{\it at} is generally used to refer to small and unimportant places like villages, small towns {\it etc. e.g., We shall meet them at the club this evening}.

\item {\it on, in, at and by (time)}

while speaking about time at indicates an exact point of time, on a more general point of time and in a period of time {\it e.g., We set out at dawn. I was born on May 26. The postman brought this letter in the morning.}

{\it by} is used to show the latest time at which an action will be finished {\it e.g., I shall be leaving by 6 o' clock}.

\item {\it on, upon}

{\it on} is generally used to talk about things at rest {\it e.g., He sat on a chair};\\
{\it upon} to talk about things in motion {\it e.g., He jumped upon his horse}.

\item {\it in, within (time)}

{\it in} means at the end of a certain period {\it e.g., The spacecraft will reach the moon in three days};

{\it within} means before the end of a certain period {\it e.g., The spacecraft will reach the moon within three days}.
\end{itemize}

\subsection{3. Verb Related Errors}

\subsubsection{Subject-Verb Agreement}
Every sentence must have subjects and verbs that agree in number. If the subject is singular, the verb must be singular. If the subject is plural, the verb must be plural as well.

{\it Australia *have/has participated in every Summer Olympics.}

{\it Diabetes mellitus *are/is classified separately due to a known defect.}
%{\it One of my dreams *are/is to build something that the whole world can admire.}

%{\it I *goes/go to school every day.}

%{\it She *watch/watches TV every evening.}

\subsubsection{Wrong Tense}
Verbs must reflect the correct tense such as:
\begin{itemize}[-]
\item Present Simple: S + V/Vs, Do/Does S + V? Signal words: every day, usually, never, at first, then, after, in the morning, evening, often.
\item Present Continuous: to be + Ving, are + we/you/they + Ving?, is + he/she/it + Ving?
am + I + Ving? Signal words: now, at the moment.
\item Present Perfect: S + have/has + Ved/3, Have + S + Ved/3? Signal words: twice, several times, lately, recently, yet, already, never, just, ever.
\item Present Perfect Continuous: I/we/you/they + have been + Ving, He/she/it + has been + Ving Signal words: since, for.
\item Past Simple: S + Ved/2, did + S + V1? Signal words: yesterday, last year/month/week.
\item Past Continuous: S + was/were + Ving, was/were + S + Ving? Signal words: at, when, while, at 2 o’clock.
\item Past Perfect: S + had + Ved/3, had + S + Ved/3? Signal words: by, before.
\item Past Perfect Continuous: S + had been + Ving.
\item Future Simple: S+will+V, will +S+V? Signal words: tomorrow, next week, next month, often , every day, soon.
\item Future Continuous: S + will be + Ving, will + S + be + Ving? Signal words: at, when, while, tomorrow.
\item Future Perfect: S + will have + Ved/3, Will + S + have + Ved/3? Signal words:  by, before, tomorrow.
\item Future Perfect Continuous: S + will have been + Ving, He/she/it + has been + Ving.
\end{itemize}

Example sentences where tense error should be corrected:

{\it All warring nations are commencing deliberations now.}

{\it Since 2000, Australia *is/has been among the top five medal winer countries.}

{\it By the end of 5th century their empire *collapsed/had collapsed.}
%{\it Yesterday, John *runs/run two miles.}

%{\it The customers *wait/have waited here for several hours.} 

%{\it Tomorrow, Tom *buys/will buy a laptop.} 

%{\it We drove to the lake, and Joe *dives/drove right in.}

%Last sentence has illogical verb tense shift error that occurs when the writer switches from present to past or from past to present without a good reason.

%{\it The pilot wanted *landing/to land in Dallas.} (\ref{4})

\subsubsection{Wrong Verb Form}
Wrong verb form error should be identified and corrected when an incorrect verb form is used {\it e.g.,} -ed ending added to irregular verb, an incorrect verb form is used for the particular tense as shown in example sentences below:

{\it They grew up together and *remain/remained friends forever.}\\
%{\it The girl  *swimmed/swam by herself.}

%{\it John *cutted/cut his finger.}

%{\it My friends just *went/have gone home.}\\
%\subsubsection{Infinitive/Gerund Confusion}

There is a list of verbs that must be followed by infinitive [\ref{12}] or gerund [\ref{13}]. The infinitive/gerund confusion error should be corrected if an incorrect form of verb is used as shown in example sentences below:

{\it They started *interfering/to interfere into disputes between local kings.}

{\it Baiju Baavera's songs helped Rafi *establish/to establish himself as a mainstream singer.}

{\it They have started *allowing/to allow Rafi *to sing/singing for them.}

{\it India started *carving/to carve its own place in the world economy.}

%{\it I enjoy *to eat/eating chocolate very much.}

%{\it I want *finishing/to finish the homework quickly.}

\subsection{Other Errors}

\subsubsection{Wrong Word Form}
If an incorrect form of the word is used {\it e.g.,} noun form is used instead of adjectival form, noun form is used instead of verb form, adjectival form is used instead of adverbial form {\it etc.} then it should be corrected as shown in example sentences below:

{\it The important *uses/usage of PHP.}

{\it Narayini explanations in Bagavath Gita is very *help/helpful for people.} 
%{\it I want to *success/succeed in life.}

%{\it This is a *China/Chinese book.}

\subsubsection{Adjective Related Errors}

If you are using more than one adjective, they usually follow a specific order: 1) article, 2) judgment, 3) size, 4) shape, 5) age, 6) color, 7) nationality, and 8) material. 

{\it He bought a *white new/new white bike.}

%\subsubsection{Misplaced Modifiers}
%The modifier should clearly refer to a specific word in the sentence.

{%\it *At eight years old/When I was eight years old, my father gave me a pony for Christmas.}

\subsubsection{Sentence Fragment}
A sentence fragment lacks a subject, a verb, or both. It cannot stand alone as a sentence. 

%{\it The dance troupe that visited our campus */was inspiring.}

%{\it The film */was very interesting.}
{\it *Very simple in use/It is very simple in use and very powerful}.

{\it *Was born/He was born in the city of Mecca on April 20th.}

{\it Blogs *are considered as a part of an open source politics.}


\subsubsection{Awkward Sentences in English}
This error involves insertion, deletion, or replacement of a single word or a phrase which is more appropriate. Also, sentence may be two wordy or the information in the sentence may be repetitive.

{\it It does not *need/require any compilation.}

{\it Japan sent *personnel/people in order to gain the further involvement.}

{\it Britain and USA became suspicious *that/due to the fact that Japan could construct permanent bases there.}
%{\it I *made my mind/decided to improve my writing skills.}

\subsection{Corpora}
\begin{itemize}[-]
\item The List of Learner Corpora - \url{http://www.uclouvain.be/en-cecl-lcWorld.html}

\item English Taiwan Learner Corpus- \url{http://www.iwillnow.org/}
\begin{itemize}
\item contains 1.5 million words of writings by Thai learners of English;
\item all materials were taken from university entrance exams at the Institute for English Language Education (IELE, Assumption University) as well as from writings by undergraduate students at various stages of their EFL course.
\end{itemize}

\item Cambridge Learner Corpus -\url{www.cambridge.org/elt/corpus/learner_corpus2.htm}
\item Chinese Learners of English Corpus (CLEC)

\item Chinese Learners of English in Hong Kong (HKUST)
\begin{itemize}
\item contains 25 million words of essays and exam scripts of upper-secondary and tertiary-level Chinese learners of English in Hong Kong (Cantonese speakers);
\item is partly tagged for part of speech and learner errors.
\end{itemize}

\item The International Corpus of Learner English (ICLE) - \url{http://www.uclouvain.be/en-cecl.html}
\begin{itemize}
\item contains argumentative essays written by advanced learners of English, i.e. university students of English as a foreign language (EFL) in their 3rd or 4th year of study. 
\end{itemize}

\item Japanese EFL Learner Corpus (JEFLL) - \url{http://www.corpora.jp/~scn/information.html?page=top}
\item Standard Speaking Test Corpus (SST) - \url{www.alc.co.jp/edusys/refecorpus.htm}
\item Thai English Learner Corpus (TELC) - \url{http://iele.au.edu/corpus/}
\item The Uppsala Student English Corpus (USE)
\item The Polish Learner English Corpus 
\item Janus Pannonius University Learner Corpus (JPU)
\end{itemize}

\subsection{Software}
\begin{itemize}
\item Pearson - \url{http://www.pearsonkt.com/prodWTL.shtml}
\item Daedalus Integrated Writing Environment - \url{www.daedalus.com/downloads.asp}
\item Internet Writing Resource for the Innovative Teaching of English (iWRITE) - \url{http://iwrite.engl.iastate.edu} (need to write an email to obtain the link, volkerh@iastate.edu)
\item CrossCheck - a grammar checker for second language writers of Swedish - \url{http://www.csc.kth.se/tcs/projects/xcheck/index-en.html}
\item Ginger -\url{http://www.gingersoftware.com/solutions/english_writing.html}
\item {\bf ETS Tools:}\\
Writing in English - \url{http://store.ets.org/store/ets/en_US/pd/productID.202264000/categoryId.11281300}\\
ProofWriter - \url{https://proofwriter.ets.org/}

\item FreeText - \url{http://129.194.19.89:8001/}
\end{itemize}

\nocite{Albert09_Annotating, Baldwin09_Pepositions, Brill00_Improved, BrockettDG06_Correcting, Burstein03_Criterion, Burstein04_Automated, Chodorow00_Unsupervised, Chodorow07_Detection, ChodorowL00_Unsupervised, Cucerzanbrill04_Spelling, DeFelice07_Automatically, DeFelice08_Classifier, Dickinson08_Developing, Dickinson10_Building, Dickinson10_Generating, Eeg-olofsson03automaticgrammar, Foster09_GenERRate, Futagi10_Effects, Gamon_usingcontextual, Gamon09_Computer, Gamon10_2_Using, Gamon10_Search,Gamon10_Using, Granfeldt05direktprofil, Han06_Detecting, Hermet09_Using, Izumi05_Error, Jones97_Contextual, Knutsson03_Transforming, Knutsson07_Designing, Koppel05_Determining, Leacock03_C-rater,  Leacock10_Automated, Lee04_Automatic, Lee07_Detection, Lee09_Human, Lee11_Building, Lonsdale03_Automated, Milton10_Toolkit, Murata04_English, Nagata06_Feedback, Omar09_Automated, Ototake09_English, Rozovskaya10_Annotating, Rozovskaya10_Generating, RozovskayaR10_Training, Schneider98_Recognizing, Shih00_compilingtaiwanese, Sun10_Learning, Tetreault08_Native, Tetreault09_Examining, Tetreault10_Reranking, Tetreault10_Using, TetreaultChodorow08_Ups, Toutanova02_Pronunciation, Tsao09_Method, Tsao09_Method, Tsur07_Using, Turner07_Language, Vandeventer01_Creating, Wible01_Web, Yi08_Web-based} 

\bibliographystyle{IEEEtran}
\bibliography{eslref}

\end{document}
